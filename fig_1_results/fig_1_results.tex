% Options for packages loaded elsewhere
\PassOptionsToPackage{unicode}{hyperref}
\PassOptionsToPackage{hyphens}{url}
%
\documentclass[
]{article}
\usepackage{amsmath,amssymb}
\usepackage{iftex}
\ifPDFTeX
  \usepackage[T1]{fontenc}
  \usepackage[utf8]{inputenc}
  \usepackage{textcomp} % provide euro and other symbols
\else % if luatex or xetex
  \usepackage{unicode-math} % this also loads fontspec
  \defaultfontfeatures{Scale=MatchLowercase}
  \defaultfontfeatures[\rmfamily]{Ligatures=TeX,Scale=1}
\fi
\usepackage{lmodern}
\ifPDFTeX\else
  % xetex/luatex font selection
\fi
% Use upquote if available, for straight quotes in verbatim environments
\IfFileExists{upquote.sty}{\usepackage{upquote}}{}
\IfFileExists{microtype.sty}{% use microtype if available
  \usepackage[]{microtype}
  \UseMicrotypeSet[protrusion]{basicmath} % disable protrusion for tt fonts
}{}
\makeatletter
\@ifundefined{KOMAClassName}{% if non-KOMA class
  \IfFileExists{parskip.sty}{%
    \usepackage{parskip}
  }{% else
    \setlength{\parindent}{0pt}
    \setlength{\parskip}{6pt plus 2pt minus 1pt}}
}{% if KOMA class
  \KOMAoptions{parskip=half}}
\makeatother
\usepackage{xcolor}
\usepackage[margin=1in]{geometry}
\usepackage{color}
\usepackage{fancyvrb}
\newcommand{\VerbBar}{|}
\newcommand{\VERB}{\Verb[commandchars=\\\{\}]}
\DefineVerbatimEnvironment{Highlighting}{Verbatim}{commandchars=\\\{\}}
% Add ',fontsize=\small' for more characters per line
\usepackage{framed}
\definecolor{shadecolor}{RGB}{248,248,248}
\newenvironment{Shaded}{\begin{snugshade}}{\end{snugshade}}
\newcommand{\AlertTok}[1]{\textcolor[rgb]{0.94,0.16,0.16}{#1}}
\newcommand{\AnnotationTok}[1]{\textcolor[rgb]{0.56,0.35,0.01}{\textbf{\textit{#1}}}}
\newcommand{\AttributeTok}[1]{\textcolor[rgb]{0.13,0.29,0.53}{#1}}
\newcommand{\BaseNTok}[1]{\textcolor[rgb]{0.00,0.00,0.81}{#1}}
\newcommand{\BuiltInTok}[1]{#1}
\newcommand{\CharTok}[1]{\textcolor[rgb]{0.31,0.60,0.02}{#1}}
\newcommand{\CommentTok}[1]{\textcolor[rgb]{0.56,0.35,0.01}{\textit{#1}}}
\newcommand{\CommentVarTok}[1]{\textcolor[rgb]{0.56,0.35,0.01}{\textbf{\textit{#1}}}}
\newcommand{\ConstantTok}[1]{\textcolor[rgb]{0.56,0.35,0.01}{#1}}
\newcommand{\ControlFlowTok}[1]{\textcolor[rgb]{0.13,0.29,0.53}{\textbf{#1}}}
\newcommand{\DataTypeTok}[1]{\textcolor[rgb]{0.13,0.29,0.53}{#1}}
\newcommand{\DecValTok}[1]{\textcolor[rgb]{0.00,0.00,0.81}{#1}}
\newcommand{\DocumentationTok}[1]{\textcolor[rgb]{0.56,0.35,0.01}{\textbf{\textit{#1}}}}
\newcommand{\ErrorTok}[1]{\textcolor[rgb]{0.64,0.00,0.00}{\textbf{#1}}}
\newcommand{\ExtensionTok}[1]{#1}
\newcommand{\FloatTok}[1]{\textcolor[rgb]{0.00,0.00,0.81}{#1}}
\newcommand{\FunctionTok}[1]{\textcolor[rgb]{0.13,0.29,0.53}{\textbf{#1}}}
\newcommand{\ImportTok}[1]{#1}
\newcommand{\InformationTok}[1]{\textcolor[rgb]{0.56,0.35,0.01}{\textbf{\textit{#1}}}}
\newcommand{\KeywordTok}[1]{\textcolor[rgb]{0.13,0.29,0.53}{\textbf{#1}}}
\newcommand{\NormalTok}[1]{#1}
\newcommand{\OperatorTok}[1]{\textcolor[rgb]{0.81,0.36,0.00}{\textbf{#1}}}
\newcommand{\OtherTok}[1]{\textcolor[rgb]{0.56,0.35,0.01}{#1}}
\newcommand{\PreprocessorTok}[1]{\textcolor[rgb]{0.56,0.35,0.01}{\textit{#1}}}
\newcommand{\RegionMarkerTok}[1]{#1}
\newcommand{\SpecialCharTok}[1]{\textcolor[rgb]{0.81,0.36,0.00}{\textbf{#1}}}
\newcommand{\SpecialStringTok}[1]{\textcolor[rgb]{0.31,0.60,0.02}{#1}}
\newcommand{\StringTok}[1]{\textcolor[rgb]{0.31,0.60,0.02}{#1}}
\newcommand{\VariableTok}[1]{\textcolor[rgb]{0.00,0.00,0.00}{#1}}
\newcommand{\VerbatimStringTok}[1]{\textcolor[rgb]{0.31,0.60,0.02}{#1}}
\newcommand{\WarningTok}[1]{\textcolor[rgb]{0.56,0.35,0.01}{\textbf{\textit{#1}}}}
\usepackage{graphicx}
\makeatletter
\def\maxwidth{\ifdim\Gin@nat@width>\linewidth\linewidth\else\Gin@nat@width\fi}
\def\maxheight{\ifdim\Gin@nat@height>\textheight\textheight\else\Gin@nat@height\fi}
\makeatother
% Scale images if necessary, so that they will not overflow the page
% margins by default, and it is still possible to overwrite the defaults
% using explicit options in \includegraphics[width, height, ...]{}
\setkeys{Gin}{width=\maxwidth,height=\maxheight,keepaspectratio}
% Set default figure placement to htbp
\makeatletter
\def\fps@figure{htbp}
\makeatother
\setlength{\emergencystretch}{3em} % prevent overfull lines
\providecommand{\tightlist}{%
  \setlength{\itemsep}{0pt}\setlength{\parskip}{0pt}}
\setcounter{secnumdepth}{-\maxdimen} % remove section numbering
\ifLuaTeX
  \usepackage{selnolig}  % disable illegal ligatures
\fi
\usepackage{bookmark}
\IfFileExists{xurl.sty}{\usepackage{xurl}}{} % add URL line breaks if available
\urlstyle{same}
\hypersetup{
  pdftitle={results},
  pdfauthor={Anton Hung},
  hidelinks,
  pdfcreator={LaTeX via pandoc}}

\title{results}
\author{Anton Hung}
\date{2024-10-07}

\begin{document}
\maketitle

\begin{Shaded}
\begin{Highlighting}[]
\FunctionTok{library}\NormalTok{(dplyr)}
\end{Highlighting}
\end{Shaded}

\begin{verbatim}
## 
## Attaching package: 'dplyr'
\end{verbatim}

\begin{verbatim}
## The following objects are masked from 'package:stats':
## 
##     filter, lag
\end{verbatim}

\begin{verbatim}
## The following objects are masked from 'package:base':
## 
##     intersect, setdiff, setequal, union
\end{verbatim}

\begin{Shaded}
\begin{Highlighting}[]
\FunctionTok{library}\NormalTok{(ggplot2)}
\end{Highlighting}
\end{Shaded}

\section{Function for generating
data}\label{function-for-generating-data}

\begin{Shaded}
\begin{Highlighting}[]
\NormalTok{generate\_data }\OtherTok{\textless{}{-}} \ControlFlowTok{function}\NormalTok{(n) \{}
\NormalTok{  W1 }\OtherTok{\textless{}{-}} \FunctionTok{runif}\NormalTok{(n, }\DecValTok{0}\NormalTok{, }\DecValTok{2}\NormalTok{)}
\NormalTok{  W2\_4 }\OtherTok{\textless{}{-}} \FunctionTok{matrix}\NormalTok{(}\FunctionTok{rbinom}\NormalTok{(n }\SpecialCharTok{*} \DecValTok{3}\NormalTok{, }\DecValTok{1}\NormalTok{, }\FloatTok{0.5}\NormalTok{), }\AttributeTok{ncol =} \DecValTok{3}\NormalTok{)}
  \FunctionTok{colnames}\NormalTok{(W2\_4) }\OtherTok{\textless{}{-}} \FunctionTok{paste0}\NormalTok{(}\StringTok{"W"}\NormalTok{, }\DecValTok{2}\SpecialCharTok{:}\DecValTok{4}\NormalTok{)}
\NormalTok{  W }\OtherTok{\textless{}{-}} \FunctionTok{cbind}\NormalTok{(}\FunctionTok{data.frame}\NormalTok{(}\AttributeTok{W1 =}\NormalTok{ W1), W2\_4)}
\NormalTok{  A }\OtherTok{\textless{}{-}} \FunctionTok{g}\NormalTok{(W)}
  \CommentTok{\# A \textless{}{-} sample(c(0, 1), size = n, replace = TRUE)}
\NormalTok{  Y }\OtherTok{\textless{}{-}} \FunctionTok{Q}\NormalTok{(}\FunctionTok{cbind}\NormalTok{(W,A))}
  \FunctionTok{cbind}\NormalTok{(W, }\FunctionTok{data.frame}\NormalTok{(}\AttributeTok{A =}\NormalTok{ A), }\FunctionTok{data.frame}\NormalTok{(}\AttributeTok{Y =}\NormalTok{ Y))}
\NormalTok{\}}

\NormalTok{g }\OtherTok{\textless{}{-}} \ControlFlowTok{function}\NormalTok{(W) \{}
\NormalTok{  p }\OtherTok{\textless{}{-}} \FunctionTok{with}\NormalTok{(W, }\FunctionTok{plogis}\NormalTok{(W1 }\SpecialCharTok{+}\NormalTok{ W2}\SpecialCharTok{*}\NormalTok{W3 }\SpecialCharTok{{-}} \DecValTok{2}\SpecialCharTok{*}\NormalTok{W4))}
  \FunctionTok{rbinom}\NormalTok{(}\FunctionTok{nrow}\NormalTok{(W), }\DecValTok{1}\NormalTok{, p)}
\NormalTok{\}}

\NormalTok{Q }\OtherTok{\textless{}{-}} \ControlFlowTok{function}\NormalTok{(W\_A) \{}
  \CommentTok{\# expit(W1 + W2xW3 + W4xA {-} 3)}
\NormalTok{  p }\OtherTok{\textless{}{-}} \FunctionTok{with}\NormalTok{(W\_A, }\FunctionTok{plogis}\NormalTok{(W1 }\SpecialCharTok{+}\NormalTok{ W2}\SpecialCharTok{*}\NormalTok{W3 }\SpecialCharTok{{-}} \DecValTok{3}\NormalTok{))}
  \FunctionTok{rbinom}\NormalTok{(}\FunctionTok{nrow}\NormalTok{(W\_A), }\DecValTok{1}\NormalTok{, p)}
\NormalTok{\}}
\end{Highlighting}
\end{Shaded}

\subsubsection{Comparison of Y1 vs Y0}\label{comparison-of-y1-vs-y0}

For each of the 1000 datasets, we can compute the sample Y1 and sample
Y0. If the ATE was truly 0, the we should see an approximately 50:50
ratio for which one is larger between Y1 and Y0.

Instead, we see and approximately 85:15 ratio in favour of Y1.

\begin{Shaded}
\begin{Highlighting}[]
\NormalTok{result }\OtherTok{\textless{}{-}} \FunctionTok{numeric}\NormalTok{(}\DecValTok{1000}\NormalTok{)}
\ControlFlowTok{for}\NormalTok{ (i }\ControlFlowTok{in} \DecValTok{1}\SpecialCharTok{:}\DecValTok{1000}\NormalTok{) \{}
  \FunctionTok{set.seed}\NormalTok{(i)}
\NormalTok{  data }\OtherTok{\textless{}{-}} \FunctionTok{generate\_data}\NormalTok{(}\DecValTok{200}\NormalTok{)}
\NormalTok{  result[i] }\OtherTok{\textless{}{-}} \FunctionTok{mean}\NormalTok{(data[}\FunctionTok{which}\NormalTok{(data}\SpecialCharTok{$}\NormalTok{A}\SpecialCharTok{==}\DecValTok{0}\NormalTok{),}\StringTok{"Y"}\NormalTok{]) }\SpecialCharTok{\textgreater{}} \FunctionTok{mean}\NormalTok{(data[}\FunctionTok{which}\NormalTok{(data}\SpecialCharTok{$}\NormalTok{A}\SpecialCharTok{==}\DecValTok{1}\NormalTok{),}\StringTok{"Y"}\NormalTok{])}
\NormalTok{\}}
\FunctionTok{table}\NormalTok{(result)}
\end{Highlighting}
\end{Shaded}

\begin{verbatim}
## result
##   0   1 
## 859 141
\end{verbatim}

\subsection{Used TML3 to generate estimates
\ldots{}}\label{used-tml3-to-generate-estimates}

\section{Recreating Figure 1 from the
paper}\label{recreating-figure-1-from-the-paper}

\begin{Shaded}
\begin{Highlighting}[]
\NormalTok{results }\OtherTok{\textless{}{-}} \FunctionTok{rbind}\NormalTok{(}\FunctionTok{readRDS}\NormalTok{(}\StringTok{"/Users/anton/Library/CloudStorage/OneDrive{-}ColumbiaUniversityIrvingMedicalCenter/Q3\_2024/random{-}seed{-}effects/fig\_1\_results/seeds{-}1{-}1000\_sensitivity.rds"}\NormalTok{))}

\NormalTok{results }\OtherTok{\textless{}{-}}\NormalTok{ results }\SpecialCharTok{|\textgreater{}} 
  \FunctionTok{mutate}\NormalTok{(}\AttributeTok{ATE =}\NormalTok{ A1\_estimate }\SpecialCharTok{{-}}\NormalTok{ A0\_estimate,}
         \AttributeTok{ATE\_low =}\NormalTok{ ATE }\SpecialCharTok{{-}} \FloatTok{1.96}\SpecialCharTok{*}\FunctionTok{sqrt}\NormalTok{(A0\_se}\SpecialCharTok{\^{}}\DecValTok{2} \SpecialCharTok{+}\NormalTok{ A1\_se}\SpecialCharTok{\^{}}\DecValTok{2}\NormalTok{),}
         \AttributeTok{ATE\_high =}\NormalTok{ ATE }\SpecialCharTok{+} \FloatTok{1.96}\SpecialCharTok{*}\FunctionTok{sqrt}\NormalTok{(A0\_se}\SpecialCharTok{\^{}}\DecValTok{2} \SpecialCharTok{+}\NormalTok{ A1\_se}\SpecialCharTok{\^{}}\DecValTok{2}\NormalTok{),}
         \AttributeTok{CI\_width =}\NormalTok{ ATE\_high}\SpecialCharTok{{-}}\NormalTok{ATE\_low)}

\NormalTok{results }\OtherTok{\textless{}{-}}\NormalTok{ results }\SpecialCharTok{|\textgreater{}}
  \FunctionTok{arrange}\NormalTok{(ATE\_low) }\SpecialCharTok{|\textgreater{}}
  \FunctionTok{mutate}\NormalTok{(}\AttributeTok{rank =} \FunctionTok{row\_number}\NormalTok{())}

\FunctionTok{ggplot}\NormalTok{(results, }\FunctionTok{aes}\NormalTok{(}\AttributeTok{x =}\NormalTok{ rank, }\AttributeTok{y =}\NormalTok{ ATE)) }\SpecialCharTok{+}
  \FunctionTok{geom\_errorbar}\NormalTok{(}\FunctionTok{aes}\NormalTok{(}\AttributeTok{ymin =}\NormalTok{ ATE\_low, }\AttributeTok{ymax =}\NormalTok{ ATE\_high, }\AttributeTok{color=}\NormalTok{ATE\_low}\SpecialCharTok{\textless{}}\DecValTok{0}\NormalTok{)) }\SpecialCharTok{+}
  \FunctionTok{geom\_point}\NormalTok{() }\SpecialCharTok{+} 
  \FunctionTok{ylim}\NormalTok{(}\SpecialCharTok{{-}}\FloatTok{0.5}\NormalTok{, }\FloatTok{0.5}\NormalTok{) }\SpecialCharTok{+}
  \FunctionTok{geom\_hline}\NormalTok{(}\AttributeTok{yintercept =} \DecValTok{0}\NormalTok{, }\AttributeTok{linetype =} \StringTok{"dashed"}\NormalTok{)}
\end{Highlighting}
\end{Shaded}

\includegraphics{fig_1_results_files/figure-latex/unnamed-chunk-4-1.pdf}

\begin{Shaded}
\begin{Highlighting}[]
\FunctionTok{ggplot}\NormalTok{(results, }\FunctionTok{aes}\NormalTok{(}\AttributeTok{x =}\NormalTok{ rank, }\AttributeTok{y =}\NormalTok{ ATE)) }\SpecialCharTok{+}
  \FunctionTok{geom\_errorbar}\NormalTok{(}\FunctionTok{aes}\NormalTok{(}\AttributeTok{ymin =}\NormalTok{ ATE\_low, }\AttributeTok{ymax =}\NormalTok{ ATE\_high, }\AttributeTok{color=}\NormalTok{ATE\_low}\SpecialCharTok{\textless{}}\DecValTok{0}\NormalTok{)) }\SpecialCharTok{+}
  \FunctionTok{geom\_point}\NormalTok{() }\SpecialCharTok{+} 
  \CommentTok{\# ylim({-}0.5, 0.5) +}
  \FunctionTok{geom\_hline}\NormalTok{(}\AttributeTok{yintercept =} \DecValTok{0}\NormalTok{, }\AttributeTok{linetype =} \StringTok{"dashed"}\NormalTok{)}
\end{Highlighting}
\end{Shaded}

\includegraphics{fig_1_results_files/figure-latex/unnamed-chunk-5-1.pdf}

\subsection{How many ATE values are above
0?}\label{how-many-ate-values-are-above-0}

\begin{Shaded}
\begin{Highlighting}[]
\FunctionTok{table}\NormalTok{(results}\SpecialCharTok{$}\NormalTok{ATE }\SpecialCharTok{\textgreater{}} \DecValTok{0}\NormalTok{)}
\end{Highlighting}
\end{Shaded}

\begin{verbatim}
## 
## FALSE  TRUE 
##   404   596
\end{verbatim}

\end{document}
